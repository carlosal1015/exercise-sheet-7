\begin{frame}
	\begin{theorem}
		Para $0\leq k\leq n$ se cumple
		\begin{math}
			w^{\prime}_{n+1}
			\left(x_{k}\right)=
			\prod\limits_{\substack{j=0\\j\neq k}}^{n}
			\left(
			x_{k}-x_{j}
			\right).
		\end{math}
	\end{theorem}

	\begin{proof}
		Si
		\begin{math}
			\omega_{\alert{n+1}}
			\left(x\right)
			\overset{\text{def}}{=}
			\prod\limits_{j=0}^{\alert{n+1}-1}
			\left(
			x-x_{j}
			\right)=
			\prod\limits_{j=0}^{n}
			\left(
			x-x_{j}
			\right)
		\end{math}, entonces
		\begin{math}
			\ln
			\left(
			\omega_{n+1}
			\left(x\right)
			\right)=
			\ln
			\left(
			\prod\limits_{j=0}^{n}
			\left(
				x-x_{j}
				\right)
			\right)=
			\sum\limits_{j=0}^{n}
			\ln
			\left(
			x-x_{j}
			\right)
		\end{math}.
		Derivando,

		\begin{align*}
			{\left(
				\ln
				\left(
					\omega_{n+1}
					\left(x\right)
					\right)
			\right)}^{\prime} & =
			{\left(
			\sum\limits_{j=0}^{n}
			\ln
			\left(
				x-x_{j}
				\right)
			\right)}^{\prime}=
			\sum\limits_{j=0}^{n}
			{\left(
			\ln
			\left(
				x-x_{j}
				\right)
			\right)}^{\prime}.
			\\
			\dfrac{
				\omega^{\prime}_{n+1}\left(x\right)
			}{
				\alert{
					\omega_{n+1}\left(x\right)
				}
			}                 & =
			\sum\limits_{j=0}^{n}
			\dfrac{{\left(x-x_{j}\right)}^{\prime}}{x-x_{j}}=
			\sum\limits_{j=0}^{n}
			\dfrac{1}{x-x_{j}}.
			\\
			\Aboxed{
			w^{\prime}_{n+1}
			\left(x\right)    & =
			\alert{
				\omega_{n+1}
				\left(x\right)
			}
			\sum\limits_{j=0}^{n}
			\dfrac{1}{x-x_{j}}.
			}
			\\
			w^{\prime}_{n+1}
			\left(x\right)
			                  & =
			\alert{
				\prod\limits_{j=0}^{n}
				\left(
				x-x_{j}
				\right)
			}
			\sum\limits_{j=0}^{n}
			\dfrac{1}{x-x_{j}}=
			\prod\limits_{\substack{j=0 \\j\neq k}}^{n}
			\left(
			x-x_{j}
			\right).
		\end{align*}
		Si $x_{k}$ un punto nodal cualesquiera, donde $0\leq k\leq n$,
		entonces
		\begin{math}
			w^{\prime}_{n+1}
			\left(x_{k}\right)=
			\prod\limits_{\substack{j=0 \\j\neq k}}^{n}
			\left(
			x_{k}-x_{j}
			\right)
		\end{math}.
	\end{proof}
\end{frame}

\begin{frame}
	\begin{theorem}
		Si $\Pi_{n}f\left(x\right)$ es el polinomio de Lagrange, entonces
		\begin{math}
			\Pi_{n}f\left(x\right)=
			\sum\limits_{k=0}^{n}
			\dfrac{
				\omega_{n+1}\left(x\right)
			}{
				\left(x-x_{k}\right)
				\omega^{\prime}_{n+1}\left(x_{k}\right)
			}
			y_{k}
		\end{math}.
	\end{theorem}

	\begin{proof}
		\begin{align*}
			\sum\limits_{k=0}^{n}
			\dfrac{
				\omega_{n+1}\left(x\right)
			}{
				\left(x-x_{k}\right)
				\alert{
					\omega^{\prime}_{n+1}\left(x_{k}\right)
				}
			}
			y_{k} & =
			\sum\limits_{k=0}^{n}
			y_{k}
			\dfrac{
				\prod\limits_{j=0}^{n}
				\left(
				x-x_{j}
				\right)
			}{
				\left(x-x_{k}\right)
				\alert{
			\prod\limits_{\substack{j=0 \\j\neq k}}^{n}
					\left(
					x_{k}-x_{j}
					\right)
				}
			}
			\\
			      & =
			\sum\limits_{k=0}^{n}
			y_{k}
			\dfrac{
			\prod\limits_{\substack{j=0 \\j\neq k}}^{n}
				\left(
				x-x_{j}
				\right)
			}{
			\prod\limits_{\substack{j=0 \\j\neq k}}^{n}
				\left(
				x_{k}-x_{j}
				\right)
			}
			\\
			      & =
			\sum\limits_{k=0}^{n}
			y_{k}
			\alert{
			\prod\limits_{\substack{j=0 \\j\neq k}}^{n}
			\dfrac{
			x-x_{j}
			}{
			x_{k}-x_{j}
			}
			}                           \\
			      & =
			\sum\limits_{k=0}^{n}
			y_{k}
			\alert{
				\ell_{k}\left(x\right)
			}
			=
			\Pi_{n}
			f\left(x\right).
		\end{align*}
	\end{proof}
\end{frame}

\begin{frame}
	\begin{theorem}[Teorema de las diferencias divididas de orden superior]
		Para $0\leq k\leq n$ se cumple
		\begin{equation*}
			a_{k}=
			f\left[x_{0},\ldots,x_{k}\right]=
			\dfrac{
			f\left[x_{1},\ldots,x_{k}\right]-
			f\left[x_{0},\ldots,x_{k-1}\right]
			}{x_{k}-x_{0}}.
			% \begin{cases}
			% 	0 & \text{si} \\
			% 	1 & \text{no}
			% \end{cases}
		\end{equation*}
		La evaluación de $a_{n}$ requiere $n^{2}$ restas y
		$\dfrac{n^{2}}{2}$ divisiones.
	\end{theorem}

	% \begin{proof}
	% 	.
	% \end{proof}

	\begin{theorem}[Representación explícita de $a_{n}$]
		¿Encontrar alguna identidad entre $\omega_{k}\left(x\right)$ y
		$\ell_{k}\left(x\right)$?
		\begin{equation*}
			a_{n}=
			\sum\limits_{k=0}^{n}
			\dfrac{
				f\left(x_{k}\right)
			}{
				w^{\prime}_{n+1}
				\left(x_{k}\right)
			}.
		\end{equation*}
	\end{theorem}

	\begin{proof}
		\begin{equation*}
			\sum\limits_{k=0}^{n}
			\dfrac{
				f\left(x_{k}\right)
			}{
				\alert{
					w^{\prime}_{n+1}
					\left(x_{k}\right)
				}
			}=
			\sum\limits_{k=0}^{n}
			\dfrac{
				f\left(x_{k}\right)
			}{
				\alert{
					\prod\limits_{\substack{j=0 \\j\neq k}}^{n}
					\left(
					x_{k}-x_{j}
					\right)
				}
			}
		\end{equation*}
	\end{proof}

	% \begin{theorem}[Interpolación de Lagrange baricéntrica] % Fórmula baricéntrica
	% 	\begin{equation*}
	% 		\lambda_{k}\coloneqq
	% 		\prod\limits_{\substack{j=0               \\j\neq k}}^{n}
	% 		\dfrac{1}{x_{k}-x_{j}}=
	% 		\dfrac{1}{\prod\limits_{\substack{j=0               \\j\neq k}}^{n}x_{k}-x_{j}}
	% 	\end{equation*}
	% \end{theorem}
\end{frame}