\section{Pregunta N$^{\circ}$23\qquad Carlos Alonso Aznarán Laos}

\begin{frame}
	\frametitle{Polinomio de interpolación}

	Sean $n+1$ puntos distintos
	\begin{math}
		{
			\left\{
			\left(x_{k},y_{k}\right)
			\right\}
		}_{k=0}^{n}\subset
		\left[a,b\right]\times\mathbb{R}
	\end{math}
	y
	\begin{math}
		f\colon\left[a,b\right]\to
		\mathbb{R}
	\end{math}
	una función de modo que
	\begin{math}
		y_{k}=
		f\left(x_{k}\right)
	\end{math}
	para $0\leq k\leq n$.

	\begin{definition}[Polinomio de interpolación en la forma de Lagrange]
		\begin{equation*}
			\Pi_{n}
			f\left(x\right)\coloneqq
			% y_{0}
			% \ell_{0}\left(x\right)+
			% y_{1}
			% \ell_{1}\left(x\right)+
			% \cdots+
			% y_{n}
			% \ell_{n}\left(x\right)=
			\sum\limits_{k=0}^{n}
			y_{k}
			\ell_{k}\left(x\right),
		\end{equation*}
		donde $\ell_{k}\left(x\right)$ son los
		\emph{polinomios característicos} dados por
		\begin{math}
			\ell_{k}
			\left(x\right)\coloneq
			\prod\limits_{\substack{j=0\\j\neq k}}^{n}
			\dfrac{x-x_{j}}{x_{k}-x_{j}}
		\end{math}
		para $0\leq k\leq n$.
	\end{definition}
	% Here $\ell_{0},\ell_{1},\ldots,\ell_{n}$ are polynomials that
	% depend on the nodes $x_0, x_1, \ldots, x_n$ but not on the
	% ordinates $y_{0},y_{1},\ldots,y_{n}$.
	% Since all the ordinates could be $0$ except for a $1$ occupying the
	% $i$-th position, we see that

	% \begin{equation*}
	% 	\delta_{ij}=
	% 	p_{n}
	% 	\left(x_j\right)=
	% 	\sum_{k=0}^{n}
	% 	y_{k}
	% 	\ell_k\left(x_j\right)=
	% 	\sum_{k=0}^{n}
	% 	\delta_{ki}
	% 	\ell_{k}
	% 	\left(x_j\right)=
	% 	\ell_{i}
	% 	\left(x_j\right).
	% \end{equation*}

	% (Recall that the Kronecker delta is defined by $\delta_{k i}=1$ if
	% $k=i$ and $\delta_{k i}=0$ if $k \neq i$.)
	% We can easily arrive at a set of polynomials having this property.
	% Let us consider $\ell_{0}$.
	% It is to be a polynomial of degree $n$ that takes the value $0$ at
	% $x_{1},x_{2},\ldots,x_{n}$ and the value $1$ at $x_0$.
	% Clearly, $\ell_{0}$ must be of the form

	% \begin{equation*}
	% 	\ell_{0}
	% 	\left(x\right)=
	% 	c
	% 	\left(x-x_1\right)
	% 	\left(x-x_2\right)\cdots
	% 	\left(x-x_n\right)=
	% 	c
	% 	\prod_{j=1}^{n}
	% 	\left(x-x_j\right).
	% \end{equation*}

	% The value of $c$ is obtained by putting $x=x_{0}$, so that
	% \begin{math}
	% 	1=
	% 	c
	% 	\prod\limits_{j=1}^{n}
	% 	\left(x_{0}-x_{j}\right)
	% \end{math}
	% y
	% \begin{math}
	% 	c=
	% 	\prod\limits_{j=1}^{n}
	% 	{\left(x_{0}-x_{j}\right)}^{-1}
	% \end{math}.

	\begin{definition}[Polinomio de interpolación en la forma de Newton]
		\begin{equation*}
			\Pi_{n}
			f\left(x\right)\coloneqq
			\sum\limits_{k=0}^{n}
			a_{k}
			\omega_{k}\left(x\right),
		\end{equation*}
		donde
		\begin{itemize}
			\item

			      \begin{math}
				      a_{k}\coloneqq
				      f\left[x_{0},\ldots,x_{k}\right]
			      \end{math}
			      es la $k$-ésima diferencia dividida de Newton, y

			\item

			      \begin{math}
				      \omega_{k}
				      \left(x\right)\coloneqq
				      \prod\limits_{j=0}^{k-1}
				      \left(x-x_{j}\right)
			      \end{math}
			      es el polinomio nodal de grado $k$.
		\end{itemize}
	\end{definition}
\end{frame}

\begin{frame}
	\begin{enumerate}\setcounter{enumi}{22}
		\item

		      Using the functions $\ell_{i}$ defined in Section 6.1
		      (p. 312) and based on nodes $x_{0}, x_{1}, \ldots, x_{n}$,
		      show that for any $f$
		      \begin{equation*}
			      \sum_{k=0}^{n}
			      f\left(x_{k}\right)
			      \ell_{k}\left(x\right)=
			      \sum_{k=0}^{n}
			      f\left[x_{0},x_{1},\ldots,x_{k}\right]
			      \prod_{j=0}^{k-1}
			      \left(x-x_{j}\right).
		      \end{equation*}

		      (Continuation) Prove this formula:
		      \begin{equation*}
			      f\left[x_{0},x_{1},\ldots,x_{n}\right]=
			      \sum_{k=0}^{n}
			      f\left(x_{k}\right)
			      \prod_{\substack{j=0\\j\neq k}}^{n}
			      {\left(x_{k}-x_{j}\right)}^{-1}.
		      \end{equation*}
	\end{enumerate}
\end{frame}

\begin{frame}
	\begin{solution}
		Sean $n+1$ puntos distintos
		\begin{math}
			{
				\left\{
				\left(x_{k},y_{k}\right)
				\right\}
			}_{k=0}^{n}\subset
			\left[a,b\right]\times\mathbb{R}
		\end{math}
		y
		\begin{math}
			f\colon\left[a,b\right]\to
			\mathbb{R}
		\end{math}
		una función de modo que
		\begin{math}
			y_{k}=
			f\left(x_{k}\right)
		\end{math}
		para $0\leq k\leq n$.

		\begin{align*}
			\sum_{k=0}^{n}
			f\left(x_{k}\right)
			\alert{\ell_{k}\left(x\right)} & =
			\sum_{k=0}^{n}
			y_{k}
			\alert{
			\prod_{\substack{j=0               \\j\neq k}}^{n}
			\dfrac{x-x_{j}}{x_{k}-x_{j}}.
			}                              & =
			\\
		\end{align*}
	\end{solution}
\end{frame}