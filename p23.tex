\section{Pregunta N$^{\circ}$23\qquad Carlos Alonso Aznarán Laos}

\begin{frame}
	\frametitle{Polinomio de interpolación}

	Sean $n+1$ puntos distintos
	\begin{math}
		{
			\left\{
			\left(x_{k},y_{k}\right)
			\right\}
		}_{k=0}^{n}\subset
		\left[a,b\right]\times\mathbb{R}
	\end{math}
	y
	\begin{math}
		f\colon\left[a,b\right]\to
		\mathbb{R}
	\end{math}
	una función de modo que
	\begin{math}
		y_{k}=
		f\left(x_{k}\right)
	\end{math}
	para $0\leq k\leq n$.

	\begin{definition}[Polinomio de interpolación en la forma de Lagrange]
		\begin{equation*}
			\Pi_{n}
			f\left(x\right)\coloneqq
			% y_{0}
			% \ell_{0}\left(x\right)+
			% y_{1}
			% \ell_{1}\left(x\right)+
			% \cdots+
			% y_{n}
			% \ell_{n}\left(x\right)=
			\sum\limits_{k=0}^{n}
			y_{k}
			\ell_{k}\left(x\right)\in\mathbb{P}_{n},
		\end{equation*}
		donde
		\begin{math}
			\ell_{k}
			\left(x\right)\coloneq
			\prod\limits_{\substack{j=0\\j\neq k}}^{n}
			\dfrac{x-x_{j}}{x_{k}-x_{j}}
		\end{math}
		para $0\leq k\leq n$ son los \alert{polinomios característicos}
		que satisface
		\begin{math}
			\ell_{k}
			\left(x_{j}\right)=
			\delta_{kj}
		\end{math}.

		La evaluación de $\Pi_{n}f\left(x\right)$ requiere
		$O\left(n^{2}\right)$ sumas y productos, en general el algoritmo
		es \emph{numéricamente inestable}.
	\end{definition}
	% Here $\ell_{0},\ell_{1},\ldots,\ell_{n}$ are polynomials that
	% depend on the nodes $x_0, x_1, \ldots, x_n$ but not on the
	% ordinates $y_{0},y_{1},\ldots,y_{n}$.
	% Since all the ordinates could be $0$ except for a $1$ occupying the
	% $i$-th position, we see that

	% \begin{equation*}
	% 	\delta_{ij}=
	% 	p_{n}
	% 	\left(x_j\right)=
	% 	\sum\limits_{k=0}^{n}
	% 	y_{k}
	% 	\ell_k\left(x_j\right)=
	% 	\sum\limits_{k=0}^{n}
	% 	\delta_{ki}
	% 	\ell_{k}
	% 	\left(x_j\right)=
	% 	\ell_{i}
	% 	\left(x_j\right).
	% \end{equation*}

	% (Recall that the Kronecker delta is defined by $\delta_{k i}=1$ if
	% $k=i$ and $\delta_{k i}=0$ if $k \neq i$.)
	% We can easily arrive at a set of polynomials having this property.
	% Let us consider $\ell_{0}$.
	% It is to be a polynomial of degree $n$ that takes the value $0$ at
	% $x_{1},x_{2},\ldots,x_{n}$ and the value $1$ at $x_0$.
	% Clearly, $\ell_{0}$ must be of the form

	% \begin{equation*}
	% 	\ell_{0}
	% 	\left(x\right)=
	% 	c
	% 	\left(x-x_1\right)
	% 	\left(x-x_2\right)\cdots
	% 	\left(x-x_n\right)=
	% 	c
	% 	\prod\limits_{j=1}^{n}
	% 	\left(x-x_j\right).
	% \end{equation*}

	% The value of $c$ is obtained by putting $x=x_{0}$, so that
	% \begin{math}
	% 	1=
	% 	c
	% 	\prod\limits_{j=1}^{n}
	% 	\left(x_{0}-x_{j}\right)
	% \end{math}
	% y
	% \begin{math}
	% 	c=
	% 	\prod\limits_{j=1}^{n}
	% 	{\left(x_{0}-x_{j}\right)}^{-1}
	% \end{math}.

	\begin{definition}[Polinomio de interpolación en la forma de Newton]
		\begin{equation*}
			\Pi_{n}
			f\left(x\right)\coloneqq
			\sum\limits_{k=0}^{n}
			a_{k}
			\omega_{k}\left(x\right)\in\mathbb{P}_{n},
		\end{equation*}
		donde
		\begin{itemize}
			\item

			      \begin{math}
				      a_{k}\coloneqq
				      f\left[x_{0},\ldots,x_{k}\right]
			      \end{math}
			      es la \alert{$k$-ésima diferencia dividida de Newton}, y

			\item

			      \begin{math}
				      \omega_{k}
				      \left(x\right)\coloneqq
				      \prod\limits_{j=0}^{k-1}
				      x-x_{j}
			      \end{math}
			      es el \alert{polinomio nodal de grado $k$}.
		\end{itemize}
		La evaluación de $\Pi_{n}f\left(x\right)$ requiere
		$O\left(n\right)$.
	\end{definition}
\end{frame}

% https://people.maths.ox.ac.uk/trefethen/barycentric.pdf
\begin{frame}
	\begin{definition}[Interpolación baricéntrica de Lagrange]
		Con el fin de realizar menos operaciones en la interpolación
		polinomial de Lagrange, multiplicamos por
		\begin{math}
			\alert{
				\dfrac{1}{\omega_{n+1}\left(x\right)}
			}
		\end{math}
		y resulta
		\begin{align*}
			\alert{
				\dfrac{1}{\omega_{n+1}\left(x\right)}
			}
			\Pi_{n}f\left(x\right) & =
			\alert{
				\dfrac{1}{\omega_{n+1}\left(x\right)}
			}
			\sum\limits_{k=0}^{n}
			y_{k}
			\alert{
				\ell_{k}\left(x\right)
			}=
			\dfrac{1}{\omega_{n+1}\left(x\right)}
			\sum\limits_{k=0}^{n}
			y_{k}
			\alert{
			\prod\limits_{\substack{j=0               \\j\neq k}}^{n}
			\dfrac{x-x_{j}}{x_{k}-x_{j}}
			}.
			\\
			                       & =
			\sum\limits_{k=0}^{n}
			\left\{
			\dfrac{y_{k}}{
			\alert{
			\prod\limits_{\substack{j=0               \\j\neq k}}^{n}x_{k}-x_{j}
			}
			}
			\dfrac{
			\alert{
			\prod\limits_{\substack{j=0               \\j\neq k}}^{n}x-x_{j}
			}
			}{\omega_{n+1}\left(x\right)}
			\right\}=
			\sum\limits_{k=0}^{n}
			\left\{
			\dfrac{y_{k}}{\prod\limits_{\substack{j=0 \\j\neq k}}^{n}x_{k}-x_{j}}
			\dfrac{1}{x-x_{k}}
			\right\}.
			\\
			\Pi_{n}f\left(x\right)
			                       & =
			\sum\limits_{k=0}^{n}
			y_{k}
			\widetilde{\ell}_{k}\left(x\right)\in\mathbb{P}_{n},
		\end{align*}
		donde
		\begin{columns}
			\begin{column}{.45\paperwidth}
				\begin{itemize}
					\item

					      \begin{math}
						      \widetilde{\ell}_{k}
						      \left(x\right)=
						      \omega_{n+1}
						      \left(x\right)
						      \dfrac{b_{k}}{x-x_{k}}
					      \end{math}, y
				\end{itemize}
			\end{column}
			\begin{column}{.45\paperwidth}
				\begin{itemize}
					\item

					      \begin{math}
						      b_{j}=
						      \dfrac{1}{
						      \prod\limits_{\substack{j=0\\j\neq k}}^{n}
						      x_{k}-x_{j}
						      }
					      \end{math}
					      son los \emph{pesos baricéntricos}.
				\end{itemize}
			\end{column}
		\end{columns}
		% La evaluación de $\Pi_{n}f\left(x\right)$ requiere
		% $O\left(n\right)$.
	\end{definition}
\end{frame}
\begin{frame}
	\begin{theorem}
		Para $0\leq k\leq n$ se cumple
		\begin{math}
			w^{\prime}_{n+1}
			\left(x_{k}\right)=
			\prod\limits_{\substack{j=0\\j\neq k}}^{n}
			x_{k}-x_{j}.
		\end{math}
	\end{theorem}

	\begin{proof}
		Si
		\begin{math}
			\omega_{\alert{n+1}}
			\left(x\right)
			\overset{\text{def}}{=}
			\prod\limits_{j=0}^{\alert{n+1}-1}
			x-x_{j}=
			\prod\limits_{j=0}^{n}
			x-x_{j}
		\end{math}, entonces
		\begin{math}
			\ln
			\left(
			\omega_{n+1}
			\left(x\right)
			\right)=
			\ln
			\left(
			\prod\limits_{j=0}^{n}
			x-x_{j}
			\right)=
			\sum\limits_{j=0}^{n}
			\ln
			\left(
			x-x_{j}
			\right)
		\end{math}.
		Derivando,

		\begin{align*}
			{\left(
				\ln
				\left(
					\omega_{n+1}
					\left(x\right)
					\right)
			\right)}^{\prime} & =
			{\left(
			\sum\limits_{j=0}^{n}
			\ln
			\left(
				x-x_{j}
				\right)
			\right)}^{\prime}=
			\sum\limits_{j=0}^{n}
			{\left(
			\ln
			\left(
				x-x_{j}
				\right)
			\right)}^{\prime}.
			\\
			\dfrac{
				\omega^{\prime}_{n+1}\left(x\right)
			}{
				\alert{
					\omega_{n+1}\left(x\right)
				}
			}                 & =
			\sum\limits_{j=0}^{n}
			\dfrac{{\left(x-x_{j}\right)}^{\prime}}{x-x_{j}}=
			\sum\limits_{j=0}^{n}
			\dfrac{1}{x-x_{j}}.
			\\
			\Aboxed{
			w^{\prime}_{n+1}
			\left(x\right)    & =
			\alert{
				\omega_{n+1}
				\left(x\right)
			}
			\sum\limits_{j=0}^{n}
			\dfrac{1}{x-x_{j}}.
			}
			\\
			w^{\prime}_{n+1}
			\left(x\right)
			                  & =
			\alert{
				\prod\limits_{j=0}^{n}
				\left(
				x-x_{j}
				\right)
			}
			\sum\limits_{j=0}^{n}
			\dfrac{1}{x-x_{j}}=
			\prod\limits_{\substack{j=0 \\j\neq k}}^{n}
			x-x_{j}.
		\end{align*}
		Si $x_{k}$ un punto nodal cualesquiera, donde $0\leq k\leq n$,
		entonces
		\begin{math}
			w^{\prime}_{n+1}
			\left(x_{k}\right)=
			\prod\limits_{\substack{j=0 \\j\neq k}}^{n}
			x_{k}-x_{j}
		\end{math}.
	\end{proof}
\end{frame}

\begin{frame}
	\begin{theorem}
		Si $\Pi_{n}f\left(x\right)$ es el polinomio de Lagrange, entonces
		\begin{math}
			\Pi_{n}f\left(x\right)=
			\sum\limits_{k=0}^{n}
			\dfrac{
				\omega_{n+1}\left(x\right)
			}{
				\left(x-x_{k}\right)\omega^{\prime}_{n+1}\left(x_{k}\right)
			}
			y_{k}
		\end{math}.
	\end{theorem}

	\begin{proof}
		\begin{equation*}
			\sum\limits_{k=0}^{n}
			\dfrac{
				\omega_{n+1}\left(x\right)
			}{
				\left(x-x_{k}\right)
				\alert{
					\omega^{\prime}_{n+1}\left(x_{k}\right)
				}
			}
			y_{k}=
			\sum\limits_{k=0}^{n}
			y_{k}
			\dfrac{
			\prod\limits_{j=0}^{n}
			x-x_{j}
			}{
			\left(x-x_{k}\right)
			\alert{
			\prod\limits_{\substack{j=0 \\j\neq k}}^{n}
			x_{k}-x_{j}
			}
			}=
			\sum\limits_{k=0}^{n}
			y_{k}
			\dfrac{
			\prod\limits_{\substack{j=0 \\j\neq k}}^{n}
			x-x_{j}
			}{
			\prod\limits_{\substack{j=0 \\j\neq k}}^{n}
			x_{k}-x_{j}
			}
			=
			\sum\limits_{k=0}^{n}
			y_{k}
			\alert{
			\prod\limits_{\substack{j=0 \\j\neq k}}^{n}
			\dfrac{
			x-x_{j}
			}{
			x_{k}-x_{j}
			}
			}
			=
			\sum\limits_{k=0}^{n}
			y_{k}
			\alert{
				\ell_{k}\left(x\right)
			}=
			\Pi_{n}
			f\left(x\right).
		\end{equation*}
	\end{proof}
\end{frame}

\begin{frame}
	\begin{theorem}[Teorema de las diferencias divididas de orden superior]
		Para $0\leq k\leq n$ se cumple
		\begin{equation*}
			a_{k}=
			f\left[x_{0},\ldots,x_{k}\right]=
			\dfrac{
			f\left[x_{1},\ldots,x_{k}\right]-
			f\left[x_{0},\ldots,x_{k-1}\right]
			}{x_{k}-x_{0}}.
			% \begin{cases}
			% 	0 & \text{si} \\
			% 	1 & \text{no}
			% \end{cases}
		\end{equation*}
		La evaluación de $a_{n}$ requiere $n^{2}$ restas y
		$\dfrac{n^{2}}{2}$ divisiones.
	\end{theorem}

	% \begin{proof}
	% 	.
	% \end{proof}

	\begin{theorem}[Representación explícita de $a_{n}$]
		¿Encontrar alguna identidad entre $\omega_{k}\left(x\right)$ y
		$\ell_{k}\left(x\right)$?
		\begin{equation*}
			a_{n}=
			\sum\limits_{k=0}^{n}
			\dfrac{
				f\left(x_{k}\right)
			}{
				w^{\prime}_{n+1}
				\left(x_{k}\right)
			}.
		\end{equation*}
	\end{theorem}

	\begin{proof}
		\begin{equation*}
			\sum\limits_{k=0}^{n}
			\dfrac{
				f\left(x_{k}\right)
			}{
				\alert{
					w^{\prime}_{n+1}
					\left(x_{k}\right)
				}
			}=
			\sum\limits_{k=0}^{n}
			\dfrac{
			f\left(x_{k}\right)
			}{
			\prod\limits_{\substack{j=0 \\j\neq k}}^{n}
			x_{k}-x_{j}
			}
		\end{equation*}
	\end{proof}

	% \begin{theorem}[Interpolación de Lagrange baricéntrica] % Fórmula baricéntrica
	% 	\begin{equation*}
	% 		\lambda_{k}\coloneqq
	% 		\prod\limits_{\substack{j=0               \\j\neq k}}^{n}
	% 		\dfrac{1}{x_{k}-x_{j}}=
	% 		\dfrac{1}{\prod\limits_{\substack{j=0               \\j\neq k}}^{n}x_{k}-x_{j}}
	% 	\end{equation*}
	% \end{theorem}
\end{frame}

\begin{frame}
	\begin{enumerate}\setcounter{enumi}{22}
		\item

		      Use la unicidad de la interpolación polinomial para
		      concluir que

		      % Using the functions $\ell_{i}$ defined in Section 6.1
		      % (p. 312) and based on nodes $x_{0}, x_{1}, \ldots, x_{n}$,
		      % show that for any $f$

		      \begin{equation*}
			      \sum\limits_{k=0}^{n}
			      f\left(x_{k}\right)
			      \ell_{k}\left(x\right)=
			      \sum\limits_{k=0}^{n}
			      f\left[x_{0},x_{1},\ldots,x_{k}\right]
			      \prod\limits_{j=0}^{k-1}
			      \left(x-x_{j}\right).
		      \end{equation*}

		      % (Continuation) Prove this formula:
		      y pruebe que la siguiente fórmula explícita es válida para
		      la $n$-ésima diferencia dividida:

		      \begin{equation*}
			      f\left[x_{0},x_{1},\ldots,x_{n}\right]=
			      \sum\limits_{k=0}^{n}
			      f\left(x_{k}\right)
			      \prod\limits_{\substack{j=0\\j\neq k}}^{n}
			      {\left(x_{k}-x_{j}\right)}^{-1}.
		      \end{equation*}
	\end{enumerate}

	\begin{proof}
		Mencionaremos el siguiente teorema para justificar la unicidad
		del polinomio de interpolación.

		\

		\begin{theorem}[Existencia y unicidad del polinomio de interpolación $\Pi_{n}$]
			Dados $n+1$ \alert{nodos distintos}
			\begin{math}
				x_{0},\ldots,x_{n}
			\end{math}
			y $n+1$ valores correspondientes
			\begin{math}
				y_{0},\ldots,y_{n}
			\end{math},
			entonces existe un \alert{único polinomio}
			$\Pi_{n}\in\mathbb{P}_{n}$ tal que para cualquier
			$0\leq k\leq n$ se cumple $\Pi_{n}\left(x_{k}\right)=y_{k}$.
		\end{theorem}

		\

		Por hipótesis, se tiene exactamente $n+1$ puntos distintos
		\begin{math}
			{
				\left\{
				\left(x_{k},y_{k}\right)
				\right\}
			}_{k=0}^{n}\subset
			\left[a,b\right]\times\mathbb{R}
		\end{math}
		y
		\begin{math}
			f\colon\left[a,b\right]\to
			\mathbb{R}
		\end{math}
		una función de modo que
		\begin{math}
			y_{k}=
			f\left(x_{k}\right)
		\end{math}
		para $0\leq k\leq n$.

		\

		Ahora, el lado izquierdo de la igualdad es el polinomio de
		interpolación en la forma de \alert{Lagrange}
		y el lado derecho es polinomio de interpolación en la forma de
		\alert{Newton}.
	\end{proof}
\end{frame}

\begin{frame}
	\begin{solution}
		Para encontrar la fórmula explícita de de la $n$-ésima forma divida

		\begin{align*}
			\sum\limits_{k=0}^{n}
			f\left(x_{k}\right)
			\alert{\ell_{k}\left(x\right)} & =
			\sum\limits_{k=0}^{n}
			y_{k}
			\alert{
			\prod\limits_{\substack{j=0        \\j\neq k}}^{n}
			\dfrac{x-x_{j}}{x_{k}-x_{j}}
			}
			\shortintertext{por el teorema X}
			                               & =
			\sum\limits_{k=0}^{n}
			a_{k}
			\omega_{k}\left(x\right).
		\end{align*}

		\begin{align*}
			\sum\limits_{k=0}^{n}
			f\left(x_{k}\right)
			\prod\limits_{\substack{j=0 \\j\neq k}}^{n}
			{\left(x_{k}-x_{j}\right)}^{-1}
			 & =
			\shortintertext{por el teorema Y}
			 & =
			a_{k}
		\end{align*}
	\end{solution}
\end{frame}

\begin{frame}
	Partiendo de la fórmula de interpolación de Lagrange y definiendo
	\begin{equation*}
		\lambda_{i}=
		\prod_{\substack{j=0 \\ j \neq i}}^{n}
		\dfrac{1}{\left(x_i-x_j\right)}=
		\dfrac{1}{\prod_{\substack{j=0 \\ j \neq i}}^{n}\left(x_i-x_j\right)}\qquad
		\mu_{i}=
		\dfrac{\lambda_{i}}{x-x_{i}}, i=0,1, \ldots, n.
	\end{equation*}
	demostrar que si $x$ no es un nodo, entonces el polinomio de
	interpolación se puede calcular mediante la fórmula:
	\begin{equation*}
		p\left(x\right)=
		\frac{\sum_{i=0}^n \mu_i y_i}{\sum_{i=0}^n \mu_i}
	\end{equation*}
	que se denomina fórmula baricéntrica del proceso de interpolación de Lagrange.
\end{frame}

\begin{frame}
	Sabemos que
	$$
		p(x)=\sum_{i=0}^n y_i \prod_{\substack{j=0 \\ j \neq i}}^n \frac{x-x_j}{x_i-x_j}=\sum_{i=0}^n y_i \frac{1}{x-x_i}\left(\prod_{\substack{j=0 \\ j \neq i}}^n \frac{1}{x_i-x_j}\right) \prod_{k=0}^n\left(x-x_k\right)
	$$
	y por otra parte
	$$
		\lambda_i=\prod_{\substack{j=0 \\ j \neq i}}^n \frac{1}{\left(x_i-x_j\right)}=\frac{1}{\prod_{\substack{j=0 \\ j \neq i}}^n\left(x_i-x_j\right)}, i=0,1, \ldots, n
	$$
	que depende solo de las abcisas $x_k$, y además
	$$
		\mu_i=\frac{\lambda_i}{x-x_i}, i=0,1, \ldots, n
	$$
	que depende del valor $x$. Con estas definiciones, (1) se puede escribir en la forma
	$$
		p(x)=\left(\sum_{i=0}^n \mu_i y_i\right) \prod_{k=0}^n\left(x-x_k\right)
	$$
	Esta última forma (4) es válida para cualquier valor de los $y_i$, en particular cuando $y_{i}=1, i=0,1, \ldots, n$. Para estos valores de la función la única solución posible es

	$p(x)=1$ por el teorema 1.1. Por tanto, aplicando (4)
	$$
		1=\left(\sum_{i=0}^n \mu_i\right) \prod_{k=0}^n\left(x-x_k\right), \quad \forall x \Longrightarrow \prod_{k=0}^n\left(x-x_k\right)=\frac{1}{\sum_{i=0}^n \mu_i}
	$$
\end{frame}

\begin{frame}
	A partir entonces de (4) y (5) deducimos
	$$
		p(x)=\frac{\sum_{i=0}^n \mu_i y_i}{\sum_{i=0}^n \mu_i}
	$$
\end{frame}