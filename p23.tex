\section{Pregunta N$^{\circ}$23\qquad Carlos Alonso Aznarán Laos}

\begin{frame}
	\frametitle{Forma de Lagrange del polinomio de interpolación}

	\begin{equation*}
		p\left(x\right)=
		y_{0}\ell_{0}\left(x\right)+
		y_{1}\ell_{1}\left(x\right)+
		\cdots+
		y_{n}\ell_{n}\left(x\right)=
		\sum\limits_{k=0}^{n}
		y_{k}\ell_{k}\left(x\right).
	\end{equation*}

	Here $\ell_{0},\ell_{1},\ldots,\ell_{n}$ are polynomials that
	depend on the nodes $x_0, x_1, \ldots, x_n$ but not on the
	ordinates $y_{0},y_{1},\ldots,y_{n}$.
	Since all the ordinates could be $0$ except for a $1$ occupying the
	$i$-th position, we see that

	\begin{equation*}
		\delta_{ij}=
		p_{n}\left(x_j\right)=
		\sum_{k=0}^{n}y_{k}\ell_k\left(x_j\right)=
		\sum_{k=0}^{n}\delta_{ki}\ell_{k}\left(x_j\right)=
		\ell_i\left(x_j\right).
	\end{equation*}

	(Recall that the Kronecker delta is defined by $\delta_{k i}=1$ if
	$k=i$ and $\delta_{k i}=0$ if $k \neq i$.)
	We can easily arrive at a set of polynomials having this property.
	Let us consider $\ell_{0}$.
	It is to be a polynomial of degree $n$ that takes the value $0$ at
	$x_{1},x_{2},\ldots,x_{n}$ and the value $1$ at $x_0$.
	Clearly, $\ell_{0}$ must be of the form

	\begin{equation*}
		\ell_{0}\left(x\right)=
		c\left(x-x_1\right)
		\left(x-x_2\right)\cdots
		\left(x-x_n\right)=
		c\prod_{j=1}^{n}
		\left(x-x_j\right).
	\end{equation*}

	The value of $c$ is obtained by putting $x=x_{0}$, so that
	\begin{math}
		1=
		c\prod\limits_{j=1}^{n}
		\left(x_{0}-x_{j}\right)
	\end{math}
	y
	\begin{math}
		c=
		\prod\limits_{j=1}^{n}
		{\left(x_{0}-x_{j}\right)}^{-1}
	\end{math}.
	Hence, we have
	\begin{equation*}
		\ell_{i}\left(x\right)=
		\prod_{\substack{j=0\\j\neq i}}^{n}
		\frac{x-x_{j}}{x_{i}-x_{j}}.
	\end{equation*}
\end{frame}

\begin{frame}
	\begin{enumerate}\setcounter{enumi}{22}
		\item

		      Using the functions $\ell_{i}$ defined in Section 6.1
		      (p. 312) and based on nodes $x_{0}, x_{1}, \ldots, x_{n}$,
		      show that for any $f$
		      \begin{equation*}
			      \sum_{i=0}^{n}f\left(x_{i}\right)\ell_{i}\left(x\right)=
			      \sum_{i=0}^{n}f\left[x_{0},x_{1},\ldots,x_{i}\right]
			      \prod_{j=0}^{i-1}\left(x-x_{j}\right).
		      \end{equation*}

		      (Continuation) Prove this formula:
		      \begin{equation*}
			      f\left[x_{0},x_{1},\ldots,x_{n}\right]=
			      \sum_{i=0}^{n}f\left(x_{i}\right)
			      \prod_{\substack{j=0\\j\neq i}}^{n}
			      {\left(x_{i}-x_{j}\right)}^{-1}.
		      \end{equation*}
	\end{enumerate}
\end{frame}