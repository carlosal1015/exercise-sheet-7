\section{Pregunta N$^{\circ}$5\qquad Khalid Zaid Izquierdo Ayllón}

\begin{frame}
	\begin{enumerate}\setcounter{enumi}{4}
		\item

		      El pentóxido de dinitrógeno gaseoso puro
		      \begin{math}
			      {N_{2}O_{5}}_{\left(\text{g}\right)}
		      \end{math}
		      reacciona en un reactor
		      intermitente según la reacción estequiométrica

		      \begin{equation*}
			      \ce{N2O5 <=> 2 N2O4 + O2}.
		      \end{equation*}

		      Calculamos la concentración de pentóxido de dinitrógeno
		      existente en ciertos instantes, obteniendo los siguientes
		      datos:

		      \begin{table}[ht!]
			      \centering
			      \begin{tabular}{|c|>{$}c<{$}|>{$}c<{$}|>{$}c<{$}|>{$}c<{$}|>{$}c<{$}|>{$}c<{$}|>{$}c<{$}|}
				      \hline
				      Tiempo (s)         & 0   & 200  & 400  & 650  & 1100 & 1900 & 2300 \\
				      \hline
				      Concentración (mm) & 5.5 & 5.04 & 4.36 & 3.45 & 2.37 & 1.32 & 0.71 \\
				      \hline
			      \end{tabular}
		      \end{table}

		      Si lo tenemos en el reactor un tiempo máximo de $35$
		      minutos ($2100$ segundos), determine la concentración de
		      pentóxido de dinitrógeno que queda sin reaccionar, usando
		      el polinomio de Taylor, Lagrange y Newton por diferencias
		      divididas implementado.
	\end{enumerate}

	\begin{solution}
		.
	\end{solution}
\end{frame}










































































































