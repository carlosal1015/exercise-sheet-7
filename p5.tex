\section{Pregunta N$^{\circ}$5\qquad Khalid Zaid Izquierdo Ayllón}

\begin{frame}
	\begin{enumerate}\setcounter{enumi}{4}
		\item

		      El pentóxido de dinitrógeno gaseoso puro
		      \begin{math}
			      {\left(
				      N_{2}O_{5}
				      \right)}_{\left(\text{g}\right)}
		      \end{math}
		      reacciona en un reactor
		      intermitente según la reacción estequiométrica
		      % \usepackage{chemformula}
		      % \ch{S + E <>[ $k_{\mathrm{SI}}$ ][ $k_{\mathrm{IS}}$ ] E.I <>[ $k_{\mathrm{PI}}$ ][ $k_{\mathrm{IP}}$ ] P + E}
		      Calculamos la concentración de pentóxido de dinitrógeno
		      existente en ciertos instantes, obteniendo los siguientes datos:

		      \begin{tabular}{c|c|c|c|c|c|c|c}
			      T (seg)       & 0   & 200  & 400  & 650  & 1100 & 1900 & 2300
			      Concentración & 5.5 & 5.04 & 4.36 & 3.45 & 2.37 & 1.32 & 0.71
		      \end{tabular}

		      Si lo tenemos en el reactor un tiempo máximo de $35$ minutos
		      ($2100$ segundos), determine la concentración de pentóxido de
		      dinitrógeno que queda sin reaccionar, usando el polinomio de
		      Taylor, Lagrange y Newton por diferencias divididas
		      implementado.
	\end{enumerate}

	\begin{solution}
		.
	\end{solution}
\end{frame}
































